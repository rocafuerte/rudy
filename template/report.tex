% Very simple template for lab reports. Most common packages are already included.
\documentclass[a4paper, 11pt]{article}
\usepackage[utf8]{inputenc} % Change according your file encoding
\usepackage{graphicx}
\usepackage{url}

%opening
\title{Report 1: Testing and analyzing the web server Rudy}
\author{Felix Leopoldo Rios}
\date{\today{}}

\begin{document}

\maketitle

\section{Introduction}

\textit{
I have made test and analyses of a small web server called rudy.
Both the code for the test and the web server are based on the code provided att the course web page.
}

%What is the main topic related to distributed systems covered in this seminar?
%Why is it important?

\section{Main problems and solutions}

At first there was some gaps in the code that one had to fill.
This had to be done in order to make the server work at all.

There where also a test provided at the course web wich was not that stable.
The problem was that the module raised an exception if one made several requests in a row.
This problem was solved by closing the tcp connection after each request.
Of course this is nothing new, but it was a problem because we thought that the provided module was stable.
However the problem still exist when one makes a huge number of requests.
To prevent that such problems one could set a limit of the number of request.

I tested the server by sending 1000 request to it and measaure the time it took for the server to handle the requests with and without simulatad delay att the server.
The tests are summarized in Table \ref{tab:results}. 
One can conclude that if the server does not delay when handeling a request, it can handle about 1000 requests in one second.
One can also conclude that the artificial delay has significance for handleing requests. 
The parsing of a request is not affected of this because this is done in another function.

For simulating that the requests comes from different machines, I started two erlang shells and ran the tests from tha both shells at the same time.
From the table one can notice that a small improvement was done by sending requests from two erlang shells. 
This is something that I can not explain for the moment.

\begin{table}[h]
\centering
\begin{tabular}{lllll}


delay (ms) & clients &  requests  & time (ms) &requests per second  \\\hline
0          & 1       &  1000      & 937       & 1.07\\\hline
40         & 1       &  1000      & 44034     & 0.02\\\hline
0          & 2       &  2000      & 1758      & 1.14\\\hline
40         & 2       &  2000      & 82153     & 0.02\\\hline


\end{tabular}
\caption{The test is done by sending 1000 requsts to the rudy server. The delay is to simulate the time it takes for the server to handle one request.}
\label{tab:results}
\end{table}

 
\section{Conclusions}


To make the server more work more efficiently handle requests it would be a good idea ta spawn the handling of each tcp connection in a new process.
If there are several clients requesting the server and the first one requests for a large amount of data, without spawning the following clients would have to wait för the first one to be handled. 
If the server spawns the requests, erlang has the ability to switch between connections and may be able to handle other requests before the first one is finished.
Moreover, if the machine has multicore support, the server could handle several requests in parallell.

If there e.g. is very heavy load on the server it could be good idea to let a different node handle the requests.



%%What have you learnt from the problem presented?
%%Was it useful?


\end{document}
