% Very simple template for lab reports. Most common packages are already included.
\documentclass[a4paper, 11pt]{article}
\usepackage[utf8]{inputenc} % Change according your file encoding
\usepackage{graphicx}
\usepackage{url}

%opening
\title{Report 1: Seminar Report Template}
\author{Felix Leopoldo Rios}
\date{\today{}}

\begin{document}

\maketitle

\section{Introduction}

\textit{
I have made test and analyses of the small web server rudy provided att the web page.
The tests are bases on the code provided att the course web page.
}

What is the main topic related to distributed systems covered in this seminar?
Why is it important?

\section{Main problems and solutions}

\textit{
At first there was some gaps in the code that one had to fill.
This had to be done in order to make the server work at all.
There where also a test provided at the course web.
This when running this this module an error was raised if one made several requests in a row.
This problem was solved by closing the tcp connection after each request.
Of course this is nothing new, but it was a problem because we thought that the provided module was stable.
However the problem still exist when one makes a huge number of requests in a row.
To prevent that such problems one could set a limit of the number of request.
One could also use the sleep function as suggested in the lab manual.



Now, how many request per second can we serve?

Is our artificial delay significant or does it disappear in the parsing overhead? 

What happens if you run the benchmarks on several machines at the same time? 

Run some tests and report your findings.






Summarize your problems, proposed solutions, etc. You do not
  need to copy\&paste your code. Only if needed, you may write down
  small code snipeds to show how you have solved a specific
  problem/question.}

Did you find any specific problem with the development of your
solution?  How did you solve it?

If you want to give a code example you can do it uing the verbatim environment.
\begin{verbatim}
this(X) ->
    Y = is(X),
    a(test(Y)).
\end{verbatim}

\section{Evaluation}

\textit{If needed, you may provide figures or tables with main results
  evaluating your proposals. For each seminar, we will provide you
  with some guidance on which kind of evaluation you should do.}


And Figures \ref{fig:results1} and \ref{fig:results2 } shows how to
add a figure with some results. These figures have been created with
gnuplot. There are tons of different kinds of plots that can be
generated with gnuplot. Make sure to check out
\url{http://gnuplot.info/demo/} and look at them so you can see what
can you do with this program.


\begin{figure}
  \begin{center}
    \includegraphics{results1.pdf}
    \caption{Some random results 1}
    \label{fig:results1}
  \end{center}
\end{figure}


\begin{figure}
  \begin{center}
    \includegraphics{results2.pdf}
    \caption{Some random results 2}
    \label{fig:results2}
  \end{center}
\end{figure}

To obtain these figures you have to:
\begin{enumerate}

\item Create the data file from the experiments (look at file
  experiment.dat)

\item Create a gnuplot file to create a figure in eps format (look at
  files results1.plot and results2.plot). These files may be very
  complex. But for the results we want to show, these examples are
  enough. To create the eps figures, execute in terminal:

\begin{verbatim}
$> gnuplot results1.plot 
\end{verbatim}

\item As pdflatex does not recognize eps files, you must convert them
  to pdf files. This is done by (it will generate a file
  results1.pdf):

\begin{verbatim}
$> epstopdf results1.eps
\end{verbatim}

\item That's it! Just include the figure as shown in this template and
  compile the latex as explained in the document ``Introduction to
  \LaTeXe''.

\end{enumerate}


If you want, you can also create a table of results as Table
\ref{tab:results}. If you look at the template code, you will see how
to do a table in \LaTeX.

\begin{table}[h]
\centering
\begin{tabular}{lcc}
First column & Second column & Third column\\\hline
Case 1 & 1.1 & 1.2\\\hline
Case 2 & 2.1 & 2.2\\\hline
Case 3 & 3.1 & 3.2\\\hline
\end{tabular}
\caption{Some random results in a table}
\label{tab:results}
\end{table}

\section{Conclusions}



\textit{Change the layout of this template as you want. It's only for
  your guidance but if you feel that you need a different structure,
  feel free to change it. The report should not be too long ($\approx$
  2-3 pages).}

To make the server more work more efficiently handle requests it would be a good idea ta spawn the handling of each tcp connection in a new process.
Yhis would make an improvement.
If there are several clients requesting the server and the first one requests for a large amount of data, without spawning the following clients would have to wait för the first one to be handled. 
If the srvern spawns the requests erlang has the ability to switch between connections and may be able to handle other requests before the first one is finished.
Morover, if the machine has multicore support, the server could handle several requests in parallell.

If there e.g. is very heavy load on the server it could be good idea to let a different node handle the requests.
This is easely mad in Erlang.


What have you learnt from the problem presented?
Was it useful?


\end{document}
